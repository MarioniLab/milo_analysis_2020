% Options for packages loaded elsewhere
\PassOptionsToPackage{unicode}{hyperref}
\PassOptionsToPackage{hyphens}{url}
%
\documentclass[
  10pt,
]{article}
\usepackage{lmodern}
\usepackage{amssymb,amsmath}
\usepackage{ifxetex,ifluatex}
\ifnum 0\ifxetex 1\fi\ifluatex 1\fi=0 % if pdftex
  \usepackage[T1]{fontenc}
  \usepackage[utf8]{inputenc}
  \usepackage{textcomp} % provide euro and other symbols
\else % if luatex or xetex
  \usepackage{unicode-math}
  \defaultfontfeatures{Scale=MatchLowercase}
  \defaultfontfeatures[\rmfamily]{Ligatures=TeX,Scale=1}
\fi
% Use upquote if available, for straight quotes in verbatim environments
\IfFileExists{upquote.sty}{\usepackage{upquote}}{}
\IfFileExists{microtype.sty}{% use microtype if available
  \usepackage[]{microtype}
  \UseMicrotypeSet[protrusion]{basicmath} % disable protrusion for tt fonts
}{}
\makeatletter
\@ifundefined{KOMAClassName}{% if non-KOMA class
  \IfFileExists{parskip.sty}{%
    \usepackage{parskip}
  }{% else
    \setlength{\parindent}{0pt}
    \setlength{\parskip}{6pt plus 2pt minus 1pt}}
}{% if KOMA class
  \KOMAoptions{parskip=half}}
\makeatother
\usepackage{xcolor}
\IfFileExists{xurl.sty}{\usepackage{xurl}}{} % add URL line breaks if available
\IfFileExists{bookmark.sty}{\usepackage{bookmark}}{\usepackage{hyperref}}
\hypersetup{
  hidelinks,
  pdfcreator={LaTeX via pandoc}}
\urlstyle{same} % disable monospaced font for URLs
\usepackage[margin=1in]{geometry}
\usepackage{longtable,booktabs}
% Correct order of tables after \paragraph or \subparagraph
\usepackage{etoolbox}
\makeatletter
\patchcmd\longtable{\par}{\if@noskipsec\mbox{}\fi\par}{}{}
\makeatother
% Allow footnotes in longtable head/foot
\IfFileExists{footnotehyper.sty}{\usepackage{footnotehyper}}{\usepackage{footnote}}
\makesavenoteenv{longtable}
\usepackage{graphicx}
\makeatletter
\def\maxwidth{\ifdim\Gin@nat@width>\linewidth\linewidth\else\Gin@nat@width\fi}
\def\maxheight{\ifdim\Gin@nat@height>\textheight\textheight\else\Gin@nat@height\fi}
\makeatother
% Scale images if necessary, so that they will not overflow the page
% margins by default, and it is still possible to overwrite the defaults
% using explicit options in \includegraphics[width, height, ...]{}
\setkeys{Gin}{width=\maxwidth,height=\maxheight,keepaspectratio}
% Set default figure placement to htbp
\makeatletter
\def\fps@figure{htbp}
\makeatother
\setlength{\emergencystretch}{3em} % prevent overfull lines
\providecommand{\tightlist}{%
  \setlength{\itemsep}{0pt}\setlength{\parskip}{0pt}}
\setcounter{secnumdepth}{5}

\author{}
\date{\vspace{-2.5em}}

\begin{document}

\pagenumbering{gobble}

\begin{figure}
\centering
\includegraphics{figs/figure1.pdf}
\caption{\label{fig:fig-1}\textbf{Detecting perturbed cell states as differentially abundant graph neighbourhoods}
(A) Schematic of the Milo workflow.
Neighbourhoods are defined on index cells, selected using a graph sampling algorithm.
Cells are quantified according to the experimental design to generate a counts table. Per-neighbourhood cell counts are modelled using a negative binomial GLM, and hypothesis testing is performed to determine differentially abundant neighbourhoods.
(B) A force-directed layout of a k-NN graph representing a simulated continuous trajectory of cells sampled from 2 experimental conditions (top panel - A: purple, B: white, bottom panel - kernel density of cells in condition `B').
(C) Hypothesis testing using Milo accurately and specifically detects differentially abundant neighbourhoods (FDR 1\%). Red points denote DA neighbourhoods.
(D)
A graph representation of the results from Milo differential abundance testing. Nodes are neighbourhoods, coloured by their log fold-change. Non-DA neighbourhoods (FDR 1\%) are coloured white, and sizes correspond to the number of cells in a neighbourhood. Graph edges depict the number of cells shared between adjacent neighbourhoods. The layout of nodes is determined by the position of the neighbourhood index cell in the force-directed embedding of single cells.}
\end{figure}










\begin{figure}
\centering
\includegraphics{figs/figure2.pdf}
\caption{\label{fig:fig-2}\textbf{Milo outperforms alternative differential abundance testing approaches}
(A) An example simulated trajectory of cells drawn from 5 groups with cells assigned to either conditions `A' (purple points) or `B' (white points). Inset bar plot shows the number of cells (y-axis) assigned to each condition according to the group from which cells were sampled (x-axis).
(B) A table describing the different methods compared to Milo, along with the input, key parameters and an overview of the testing framework for each.
(C) Rankings of DA testing methods across a number of measures to determine performance. Each box is coloured by the ranking of each measure for each method, where a rank of 1 indicates the best performance and 5 indicates the worst across 100 simulated data sets. Ranks are calculated from the mean value across 100 simulated data sets; mean values are shown. PPV: positive predictive value, NPV: negative predictive value, TNR: true negative rate, F1: F1 score, FOR: false omission rate, FNR: false negative rate, FPR: false positive rate, FDR: false discovery rate.
(D) The Matthews correlation coefficient assesses the performance of each method by integrating across multiple performance measures. Box plots show the MCC across 100 independent simulations for each method. Boxplots denote the median and interquartile range (IQR), with whiskers extending 1.5 x IQR; outliers are shown as individual points.}
\end{figure}







\begin{figure}
\centering
\includegraphics{figs/figure3.pdf}
\caption{\label{fig:fig-3}\textbf{Milo efficiently scales to large data sets}
(A) Run time (y-axis) of the Milo workflow from graph building to differential abundance testing. Each point represents a down-sampled dataset, denoted by shape. Coloured points show the total number of cells in the full dataset labelled by the elapsed system time (mins).
(B) Total memory usage (y-axis) across the Milo workflow. Each point represents a down-sampled dataset, denoted by shape. Coloured points are the full datasets labelled with the total memory usage (megabytes).}
\end{figure}





\begin{figure}
\centering
\includegraphics{figs/figure4.pdf}
\caption{\label{fig:fig-4}\textbf{Milo identifies the decline of a fate-biased precursor in the ageing mouse thymus}
(A-B) A UMAP of single thymic epithelial cells sampled from mice aged 1-52 weeks old. Points are labelled according to their annotation in Baran-Gale et al.~2020 (A) and mouse age (B)
(C) A graph representation of the results from Milo differential abundance testing. Nodes are neighbourhoods, coloured by their log fold change across ages. Non-DA neighbourhoods (FDR 10\%) are coloured white, and sizes correspond to the number of cells in a neighbourhood. Graph edges depict the number of cells shared between adjacent neighbourhoods. The layout of nodes is determined by the position of the neighbourhood index cell in the UMAP embedding of single cells.
(D) Beeswarm plot showing the distribution of log-fold change across age in neighbourhoods containing cells from different cell type clusters. DA neighbourhoods at FDR 10\% are coloured. Cell types detected as DA through clustering by Baran-Gale et al.~(2020) are annotated in the left side bar.
(E) A heatmap of genes differentially expressed between DA neighbourhoods in the Intertypical TEC cluster. Each column is a neighbourhood and rows are differentially expressed genes (FDR 1\%). Expression values for each gene are scaled between 0 and 1. The top panel denotes the neighbourhood DA log fold-change.}
\end{figure}







\begin{figure}
\centering
\includegraphics{figs/figure5_letters_2.pdf}
\caption{\label{fig:fig-5}\textbf{Milo identifies the compositional disorder in cirrhotic liver}
(A-B) UMAP embedding of 58358 cells from healthy (n = 5) and cirrhotic (n = 5) human livers. Cells are colored by cellular lineage (A) and injury condition (B)
(C) Graph representation of neighbourhoods identified by Milo.
Nodes are neighbourhoods, coloured by their log fold change between cirrhotic and healthy samples. Non-DA neighbourhoods (FDR 10\%) are coloured white, and sizes correspond to the number of cells in a neighbourhood. Graph edges depict the number of cells shared between adjacent neighbourhoods. The layout of nodes is determined by the position of the neighbourhood index cell in the UMAP embedding of single cells.}
\end{figure}






\newpage

(D) Beeswarm plot showing the distribution of log-fold change in abundance between conditions in neighbourhoods from different cell type clusters. DA neighbourhoods at FDR 10\% are coloured. Cell types detected as DA through clustering by Ramachandran et al.~(2019) are annotated in the left side bar. (E) UMAP embedding and graph representation of neighbourhoods of 7995 cells from endothelial lineage. (F) Heatmap showing average neighbourhood expression of genes differentially expressed between DA neighbourhoods in the endothelial lineage (572 genes). Expression values for each gene are scaled between 0 and 1. The top panel denotes the neighbourhood DA log fold-change.

\end{document}
